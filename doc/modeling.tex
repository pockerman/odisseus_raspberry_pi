\section{Modeling}
\label{modeling}

This section discusses the modeling approach that Odisseus follows

\subsection{Kinemtics Model}
\label{kinemtatics_model}

Odisseus is using the following kinematic model 

\begin{eqnarray}
\frac{dx}{dt} = v cos(\theta) \\
\frac{dy}{dt} = v sin(\theta) \\
\frac{d\theta}{dt} = \omega 
\end{eqnarray}

where $x,y$ are are the coordinates of the reference point, $\theta$ is the yaw angle, $v$ is the input velocity
and $\omega$ is the input angular velocity of the robotic platform.

The state vector $\mathbf{x}$ has three components; the $x, y$ components of the 
reference point and the orientation or yaw angle $\theta$. Mathematically, this is written as

\begin{equation}
\mathbf{x} = (x, y, \theta)
\label{state_vector}
\end{equation}

As mentioned previously, the velocity $v$ is one of the inputs that is given to the system. Namely, it is calculated according to

\begin{equation}
v = \frac{v_L*R + v_R *R}{2}
\label{odisseus_velocity}
\end{equation}

where $R$ is the wheels radius and $v_R,v_L$ are the right and left wheels velocities respectively. Similarly the second input to the system is the angular velocity of 
the robot which is given by

\begin{equation}
\omega = \frac{\omega_L*R + \omega_R *R}{L}
\label{odisseus_angular_velocity}
\end{equation}

where $L$ is the axle length connecting the two motorized wheels. $\omega_L, \omega_R$ are the angular velocities of the left and right wheels respectively.

This kinematic model is used as a motion model in the Extended Kalman Filter discussed in section \ref{extended_kalman_filter}. Concretely, the following discretized 
form is utilized 


\begin{eqnarray}
x_k = x_{k-1} + (\Delta t v_k + \mathbf{w}_{1,k})cos(\theta_{k-1} +  \Delta t \omega_k + \mathbf{w}_{2,k}) \\
y_k = y_{k-1} + (\Delta t v_k + \mathbf{w}_{1,k})sin(\theta_{k-1} +  \Delta t \omega_k + \mathbf{w}_{2,k}) \\
\theta_k = \theta_{k -1} + \Delta t \omega_k + \mathbf{w}_{2,k}
\label{odisseus_discrete}
\end{eqnarray}

where $\Delta t$ is the sampling rate and $\mathbf{w}$ is an error vector such that

\begin{equation}
E[\mathbf{w}] = \mathbf{0}
\end{equation}