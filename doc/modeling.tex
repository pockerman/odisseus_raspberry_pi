\section{Kinematics Model}
\label{modeling}
The Extended Kalman Filter discussed in section \ref{extended_kalman_filter}, requires
as motion model as input in order to make a predictions about the
pose of the robot. This section discusses the kinematics model used by Odisseus.

\subsection{Unicycle model}
\label{unicycle_model}

Odisseus is using the following unicycle model in order to capture the kinematics
of the robot motion 

\begin{eqnarray}
\frac{dx}{dt} = v cos(\theta) \\
\frac{dy}{dt} = v sin(\theta) \\
\frac{d\theta}{dt} = \omega
\label{continuous_kinematic_model} 
\end{eqnarray}

where $x,y$ are are the coordinates of the reference point, $\theta$ is the yaw angle, $v$ is the input velocity and $\omega$ is the input angular velocity of the robotic platform.

The state vector $\mathbf{x}$ has three components; the $x, y$ components of the 
reference point and the orientation or yaw angle $\theta$. Mathematically, this is written as

\begin{equation}
\mathbf{x} = (x, y, \theta)
\label{state_vector}
\end{equation}

As mentioned previously, the velocity $v$ is one of the inputs that is given to the system. Namely, it is calculated according to

\begin{equation}
v = \frac{v_l + v_r}{2}
\label{odisseus_velocity}
\end{equation}

where $R$ is the wheels radius and $v_r,v_l$ are the right and left wheels velocities respectively. Both are related to the angular wheel velocities $\omega_r,$ and  $\omega_l$ respectively and the wheel radius $R$ according to equation \ref{wheel_velocity}

\begin{equation}
v_i = \omega_iR, ~~ i = r, l
\label{wheel_velocity}
\end{equation}
Similarly the second input to the system is the angular velocity of 
the robot $\omega$. This is related to $v_l$ and $v_r$ according to equation \ref{odisseus_angular_velocity}

\begin{equation}
\omega = \frac{v_l - v_r}{2L}
\label{odisseus_angular_velocity}
\end{equation}

\subsection{Discrete kinematic model}
\label{discrete_kinematic model}
Equation \ref{continuous_kinematic_model} represents a continuous model. Odisseus, instead uses a discrete
counterpart of the model given by the equations below. 

\subsubsection{Case $\omega=0$}
This case translates to the situation where the heading of the robot remains the same.
In this case the model will simply update the $x$ and $y$ coordinates of the reference point 
according to the equations \ref{equ1} and \ref{equ2} respectively.

\begin{eqnarray}
x_k = x_{k-1} + (\Delta t v_k + \mathbf{w}_{1,k})cos(\theta_{k-1} + \mathbf{w}_{2,k}) \label{equ1} \\
y_k = y_{k-1} + (\Delta t v_k + \mathbf{w}_{1,k})sin(\theta_{k-1} + \mathbf{w}_{2,k}) \label{equ2}
\end{eqnarray} 

\subsubsection{Case $\omega \neq 0$}
When the $\omega$ is deemed to be non zero, then the following equations are used
in order to estimate the pose of the robot.

\begin{eqnarray}
\theta_k = \theta_{k -1} + \Delta t \omega_k + \mathbf{w}_{2,k} \\
\label{equ3}
x_k = x_{k-1} + (\frac{v_k}{2w_k} + \mathbf{w}_{1,k})(sin(\theta_k) - sin(\theta_{k-1})) \label{equ4} \\
y_k = y_{k-1} - (\frac{v_k}{2w_k} + \mathbf{w}_{1,k})(cos(\theta_{k}) - cos(\theta_{k-1})) \label{equ5}
\end{eqnarray}

Note that we first update the heading of the robot and then the $x$ and $y$ coordinates
of the reference point.

Both scenarios incorporate the error by assuming that this is additive. The error is accounted for
the linear and angular velocities. $\Delta t$ is the sampling rate. 